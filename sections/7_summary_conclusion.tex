\chapter{Conclusion}
Even though not all the tasks of the assignment were possible to accomplish with the method chosen, we think that the project was a huge success. The main goals of a professionally built, easy to use, directive and steerable loudspeaker was reached. The ultrasound measurements, in \ref{6_sec:ultrasonic}, and the human expertise tests, in \ref{6_sec:expertise_test}, objectively showed this. Additionally we think we brought new ideas into how ultrasonic phased arrays are built by utilising signal processing tricks such as the sigma delta modulation.     

\section{Continuing Work}
\todo[inline]{Add some text here...}

\begin{itemize}
		\item Beam-Focusing improvement
		\item Higher order sigma-delta-modulator
		\item Improve wireless performance with external antenna
		\item Develop Remote or app to control the Audio-Beamformer
		\item Frequency Response Measurement and compensation 
\end{itemize}

\newpage
\section{Personal Reflections}

\subsubsection{Florian Baumgartner}
This bachelor's thesis proved to by very challenging, since it covered pretty much every field of electrical engineering. This however, made it very attractive to work on the project, due to the enormous amount of variety in different topics. I personally could make use of my previously gained knowledge to accelerate the development process. It was a fantastic experience to design a fully working and professional looking product in such a small time frame. I'm very happy with the end result and hope that it will satisfy its purpose of convincing potential new students to start studying electrical engineering. It was a pleasure to work with Thierry Schwaller and we had overall a great time working on this project.

\subsubsection{Thierry Schwaller}
I am personally really proud of the things we've reached throughout this bachelor's thesis. I am convinced that the Audio-Beamformer could become a commercial product if more research is done. 
Additionally, I learned a lot during this assignment about the theory behind the inner workings of phased arrays and the difficulties in connecting theory to the real world. 
The collaboration with Florian Baumgartner helped me a lot to see and try new angles to solve a problem. 