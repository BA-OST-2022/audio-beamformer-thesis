\begin{abstract}
Naturally, sound waves propagate in an omnidirectional pattern. However, in some cases, a directional pattern would be preferred. In comparison to light or electromagnetic waves, it turns out to be a very difficult task to focus audio waves in a specific direction. The underlying reason for this behavior comes from the physical large wave length of the audible spectrum.\\
This bachelor's thesis concentrates on how to overcome this effect by using higher frequencies in the ultrasonic spectrum and therefore be able to create a highly directional audio beam. In addition, beam-steering methods are applied to change the direction and focus-point by software.

In order to achieve a directional sound beam, a linear phased array has been developed and build, consisting of 19 rows of 8 ultrasonic transducers each. For this a single PCB, consisting of over 1300 components was specially designed. Two FPGAs modulate the base band audio signal onto a 40 KHz ultrasonic carrier. First order Sigma-Delta-Modulators perform the analog conversation in combination with a Class-D amplifier output stage. In addition, each channel can be delayed and attenuated individually. A Raspberry Pi Compute Module 4 is used to apply real-time digital signal processing techniques to further improve the audio quality. Advanced face-detection algorithms are used to locate a target and therefore be able to direct the sound in its direction. As an input source, Bluetooth$^{\circledR}$ and AirPlay$^{\circledR}$ streaming is supported, as well as other input devices, such as USB-Microphones.

In a comprehensive human expertise test, the directivity, beam-steering capability and overall audio quality has been determined. The Audio Beamformer performs well in all categories and satisfies the goals of the project. Specially the large range of up to 50 meters is very impressive.\\
Overall this thesis was a huge success. With some further improvements, the Audio Beamformer can lead to a real alternative to conventional loudspeakers.
\end{abstract}

%As our thesis, we developed the Audio Beamformer, a fully functional device that satisfies all these demands. For this, we had to create the mechanical part, including a PCB, gateware on two FPGAs, and software on a Raspberry Pi. For a better user experience, we additionally created an intuitive graphical user interface. 

%To get an insight into the audio quality, beam-steering and directivity, we conducted a human expertise test, in which we tested our device with the help of 17 people.
%These tests verified the high directivity and beam-steering capabilities of the Audio Beamformer.
%The music quality was rated 4.2/6, and the quality of speech played through our device was rated even higher at 4.5/6. In our opinion, this is a huge success and could, in the future, even be improved further so that the Audio Beamformer would be a real alternative to conventional loudspeakers.

%Naturally, sound waves propagate in an omnidirectional pattern. However, in most cases, a specific target location is preferred. In comparison to light or electromagnetic waves, it turns out to be a very difficult task to focus audio aves in a specific direction. The underlying reason for this behavior comes from the physical large wave length of the audible spectrum.\\

%This bachelor's thesis concentrates on how to overcome this effect by using higher frequencies in the ultrasonic spectrum and therefore be able to create a highly directional audio beam. In addition, beam-steering methods are applied to change the direction and focus-point by software.

%In order to achieve a directional sound beam, a linear phased array has been developed, consisting of 19 rows of 8 ultrasonic transducers each. Two FPGAs modulate the base band audio signal onto a 40 KHz ultrasonic carrier. First order Sigma-Delta-Modulators perform the analog conversation in combination with a Class-D amplifier output stage. In addition, each channel can be delayed and attenuated individually. A Raspberry Pi Compute Module 4 is used to apply real-time digital signal processing techniques to further improve the audio quality. Advanced face-detection algorithms are used to locate a target and therefore be able to direct the sound in its direction. As an input source, Bluetooth$^{\circledR}$ and AirPlay$^{\circledR}$ streaming is supported, as well as other input devices, such as USB-Microphones.

%As a result, a fully working device has been developed and built. A single PCB, consisting of over 1300 components was specially designed to fulfill the needs of the project. In a comprehensive human expertise test, the directivity, beam-steering capability and overall audio quality has been determined. The Audio Beamformer performs well in all categories and satisfies the goals of the project. Specially the large range of up to 100 meters is very impressive.\\
%Overall this thesis was a huge success. With some further improvements, the Audio Beamformer can lead to a real alternative to conventional loudspeakers.